\documentclass[14pt]{beamer}
\usetheme{Pittsburgh}
\usepackage{xcolor}
\definecolor{USUBlue}{RGB}{26,57,89}
\usecolortheme[named=USUBlue]{structure}
\setbeamercolor{normal text}{fg=USUBlue}
\usepackage[utf8]{inputenc}
\usepackage{mathptmx}
\usepackage{tgbonum}
\usepackage{amssymb}
\usepackage{graphicx}
\usefonttheme{structuresmallcapsserif}
\usefonttheme{serif}
\setbeamercolor{author}{fg=USUBlue}
\setbeamerfont{author}{size=\small}
\setbeamerfont{frametitle}{size=\large}
\setbeamertemplate{enumerate items}[default]
\author{Elita Baldridge}
\title[17pt]{A data-intensive assessment of the species-abundance distribution.}
\setbeamertemplate{navigation symbols}{}
\date{}
%\setbeamercovered{transparent} 
%\logo{\includegraphics[scale=.03]{../Miscellaneous/Pictures/ecology_center_horizontal.jpg}\includegraphics[scale=0.07]{../Miscellaneous/Pictures/Weecology.png}} 
\institute{\includegraphics[scale=.07]{../Miscellaneous/Pictures/ecology_center_horizontal.jpg}\includegraphics[scale=0.1]{../Miscellaneous/Pictures/Weecology.png}} 
%\subject{} 
\begin{document}


\begin{frame}[t]
\titlepage
\end{frame}

%\begin{frame}
%\tableofcontents
%\end{frame}

\section{Introduction}
\subsection{What is macroecology?}
\begin{frame}[t]
\frametitle{Macroecology}
\normalsize One approach to studying ecological patterns and processes.\\
\begin{itemize}
\item Data intensive.
\item Large scales
\begin{itemize}
\item Spatial
\item Temporal
\item Taxonomic
\end{itemize}
\item Search for generality.
\end{itemize}
\end{frame}

\subsection{Criticisms of macroecology}
\begin{frame}[t]
\frametitle{Macroecology}
Criticisms of macroecology\\
\begin{itemize}
\item North American terrestrial bias.
\item Lack of identification of pattern generating mechanisms.
\end{itemize}
\end{frame}

\subsection{Best practices}
\begin{frame}[t]
\frametitle{Macroecology}
Best practice recommendations\\
\begin{itemize}
\begin{small}
\item Test patterns with multiple taxonomic groups/ecosystems.  
\item Simultaneous testing of competing models and model predictions with a consistent statistical approach.
\end{small}
\end{itemize}
\end{frame}

\section{Data}
\subsection{Introduction to Ecoinformatics}
\begin{frame}[t]
\frametitle{The Rules of Ecoinformatics}
\begin{Large}
Garbage in, garbage out.\\
\end{Large}
\begin{itemize}
\item All data are good, not all data are appropriate.
\item Fit the data to the question.
\end{itemize}
\end{frame}

\subsection{Current Data}
\begin{frame}[t]
\frametitle{Data}
\vspace{-7pt}
\includegraphics[scale=.55]{./sad-data/chapter3/presentation_map.png}
\end{frame}

\begin{frame}[t]{}
\frametitle{Data}
\begin{large}
Major macroecological datasets\\
\end{large}
\begin{itemize}
\item Largely terrestrial
\item Largely North American
\item Many publicly available, some not.
\end{itemize}
~\\
~\\
~\\
~\\
\begin{large}
Lots of data in the literature.\\
\end{large}
\end{frame}



\subsection{Compiled Data}
\subsubsection{Data Collection}
\begin{frame}[shrink=30]
\frametitle{Data}
\begin{table}
\begin{tabular}{l|l} 
 Variable name & Variable definitions\\ 
\hline
 Class & Taxonomic class of species \\
 Family & Taxonomic family of species \\
 Genus & Taxonomic genus of species\\
 Species & Specific epithet of species  \\
 Relative\_abundance & Relative abundance of species \\
 Abundance & Abundance of species \\
 Collection\_Year & Start of collecting \\
 End\_Collection & End of collecting \\
 Site\_Name & Name/description of site \\
 Biogeographic\_region & Biogeographic region \\
 Site\_notes & Additional site information \\ 
\end{tabular}
\caption{List of variables collected.}
\end{table}
\end{frame}

\subsubsection{Data Summary}
\begin{frame}{}
\frametitle{Data}
\includegraphics[scale=.5]{./sad-data/chapter2/bioregions.png}
\end{frame}

\begin{frame}{}
\frametitle{Data}
\includegraphics[scale=.5]{./sad-data/chapter2/taxa_sites.png}
\end{frame}

\begin{frame}{}
\frametitle{Data}
\includegraphics[scale=.5]{./sad-data/chapter2/num_taxa.png}
\end{frame}

\section{Species abundance distribution}
\begin{frame}[t]{}
\frametitle{Commonness \& rarity}
~\\
\begin{large}
The species abundance distribution:
\end{large}
\begin{itemize}
\item Describes the distribution of commonness \& rarity of species.
\item One of the most fundamental and ubiquitous patterns in ecology.
\item Exhibits a hollow curve distribution.
\begin{itemize}
\item Many rare species.
\item Few common species.
~\\
\end{itemize}
\item Many forms of the species abundance distribution (SAD).
\end{itemize}
\end{frame}

\section{Species abundance distribution comparisons}
\subsection{Analysis}
\subsection{Results}
\begin{frame}{}
\frametitle{SAD Comparisons}
\includegraphics[scale=.5]{./sad-data/chapter1/AICc_weights.png}
\end{frame}

\begin{frame}{}
\frametitle{SAD Comparisons}
\includegraphics[scale=.5]{./sad-data/chapter1/likelihoods.png}
\end{frame}

\begin{frame}{}
\frametitle{SAD Comparisons}
\includegraphics[scale=.5]{./sad-data/chapter1/likelihoods_one_to_one.png}
\end{frame}

\section{Neutral analysis}
\subsection{Analysis}
\subsection{Results}
\begin{frame}{}
\frametitle{Neutral Analysis}
\includegraphics[scale=.25]{./sad-data/chapter3/distabclasses_vs_lognormwgt.png}
\end{frame}

\begin{frame}{}
\frametitle{Neutral Analysis}
\includegraphics[scale=.6]{./sad-data/chapter3/avgvals_by_dataset.png}
\end{frame}

\section{Conclusions}

\section{Acknowledgements}
\begin{frame}[t]{}
\frametitle{Acknowledgements}
~\\ %Adds vertical space for better aesthetics
\small{Funding sources:}
\begin{tiny}
\begin{itemize}
\item USU Department of Biology
\item Intellectual Ventures private funding to Morgan Ernest
\item National Science Foundation CAREER Grant to Ethan White
\item Gordon \& Betty Moore Foundation's Data-Driven Discovery Initiative Grant to Ethan White.
\item USU Graduate School Dissertation Fellowship
\end{itemize}
\end{tiny}
\end{frame}

\begin{frame}{}
\frametitle{Acknowledgements}
Weecologists past, present, \& future\\
\includegraphics[scale=.3]{../Miscellaneous/Pictures/Phd/whiteboard.png}
\begin{tiny}
(especially Xiao Xiao \& Ken Locey (creator of the whiteboard))
\end{tiny}
\end{frame}

\subsection{Accessibility}
\begin{frame}[t]{}
\frametitle{Acknowledgements}
~\\ %Adds vertical space for better aesthetics
Dr. Thomas Price \& USU Student Health Center
\end{frame}
\end{document}